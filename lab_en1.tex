\documentclass[a4paper,14pt]{article} 
\usepackage{lscape}
%%% Страница
\usepackage{extsizes} % Возможность сделать 14-й шрифт
\usepackage{geometry} % Простой способ задавать поля
        \geometry{top=25mm}
        \geometry{bottom=30mm}
        \geometry{left=30mm}
        \geometry{right=20mm}
 %

\usepackage[T1,T2A]{fontenc}
\usepackage[utf8]{inputenc}
\usepackage{amssymb,amsmath}
\usepackage{float}
\usepackage[unicode, pdftex]{hyperref}
\usepackage[europeanresistors,americaninductors]{circuitikz}
\usetikzlibrary{calc}
\usepackage[T1,T2A]{fontenc}
\usepackage[utf8]{inputenc}

\usepackage{amssymb,amsmath}
\usepackage{float}
\usepackage[unicode, pdftex]{hyperref}

\usepackage{tikz}
\usepackage{rotating}
%\usepackage[landscape]{geometry}
\usepackage{graphicx}
\graphicspath{{pictures/}}
\DeclareGraphicsExtensions{.pdf,.png,.jpg}
\usepackage{pgfplots}
\usepackage{wrapfig}
\usepackage{rotating}
\usepackage{lipsum}
\usepackage{nccmath}
\usepackage{caption}
\usepackage{siunitx}
%\usepackage[american,cuteinductors,smartlabels]{circuitikz}
%\usepackage[backend=biber]{biblatex}

\usepackage[]{hyperref}
\ctikzset{bipoles/thickness=1}
\ctikzset{bipoles/length=0.8cm}
\ctikzset{bipoles/diode/height=.375}
\ctikzset{bipoles/diode/width=.3}
\ctikzset{tripoles/thyristor/height=.8}
%\ctikzset{tripoles/thyristor/width=1}
\ctikzset{tripoles/thyristor/width=0.8}
\ctikzset{bipoles/vsourceam/height/.initial=.7}
\ctikzset{bipoles/vsourceam/width/.initial=.7}
\tikzstyle{every node}=[font=\small]
\tikzstyle{every path}=[line width=0.8pt,line cap=round,line join=round]

\ctikzset{resistor = european}
\ctikzset{inductor = american}
\ctikzset{tripoles/thyristor/height=0.55}
\ctikzset{tripoles/thyristor/height 2=0.4}
%\ctikzset{tripoles/thyristor/width=0.35} 
\ctikzset{tripoles/thyristor/diode width left=0.35}
\ctikzset{tripoles/thyristor/diode width right=0.35}

\ctikzset{bipoles/cuteindictor/coils/.initial=3}
\ctikzset{bipoles/americanindictor/coils/.initial=3}
\ctikzset{bipoles/cuteindictor/coils/.initial=3}
\ctikzset{bipoles/americanindictor/coils/.initial=3}
\hypersetup{
colorlinks=false,
}
\usepackage{textcomp}

\begin{document}

\section{Research of characteristics of power semiconductor devices}

The main intension of the work is to give hands on experience for students 
with the classification system of power semiconductor devices (SPD),
methods of testing and experimental study of the characteristics of a thyristor.

\subsection{Lab Description}

The laboratory setup consists of two power sources for the power circuit of the SPD for drawing their current-voltage
characteristics in open and closed states, a direct current source for powering the control circuit
tested SP and protection and blocking circuits.

The lab is turned on by the switch $S1$ (fig. \ref{labsetup}), and the preparation of the junction of one of two sources
power supply circuit power supply -- by connection the $XP1$ plug connector in the socket $XS1$ or $XS2$ and turn
switch $S4$ to the appropriate position. The SPD under review (thyristor VS1) is connected to terminals $X1-X4$.

The voltage and current of the power circuit of the SPP are regulated by the T1 autotransformer. 
SPD control current adjustment
carried out by potentiometer $R1$ (roughly) and rheostat $R2$ (accurately). Values  control current and voltage
are determined by a voltmeter $PVЗ$ and an ammeter $PA3$.

The power supply circuit of the SPD under the studying of its characteristics 
in the open state is carried out through transformer $TK$. 
To do this, $XP1$ is connected to $XS2$, and switch $S4$ is set to the lower position.
The half-wave form of voltage and current of the SPD, necessary for characterization, 
is provided by the $VD5$ diode
and the sinusoidality of the current curve is achieved by the diodes $VD6$, $VD7$ 
(creating a circuit for the flow of current
in the second (non-working) half-period of the transformer voltage $T3$ and excluding due to this bias
of its core by direct current) and inductance of the inductor $L$, 
(limiting the value and improving the current transformer shape). 
The average values of voltage and current of the SPD are determined by a $PV2$ voltmeter
and ammeter $PA2$. To observe the curves of the anode voltage and current, 
the input of the oscilloscope is connected
to terminals $X8-X9$ and $X8-X10$, respectively.

When examining the characteristics of the SPD in the closed state, $XP1$ is connected to $XS1$, and switch $SS4$
is installed in the upper position, in which the power circuit of the $SPD$ is supplied  through
transformer $T2$. 
The average values of the voltage and current of SPD are measured with a $PVI$ voltmeter and a $PAI$ ammeter.
The polarity of the input voltage is determined by the position of the switch $S3$. 
Classification characteristics is observed with the switch $S2$ open. 
In this case, to the SPD through the diodes $VD1$, $VD2$, $VDЗ$ is supplied
half-wave voltage from the secondary winding of the transformer $T2$. 
The $VD4$ diode sharply reduces the value
negative half-wave voltage applied to the diode $VD3$ and SPD, which ensures the flow of current
through SPD in only one direction. To observe the shape of the curves of the anode voltage and current input
the oscilloscope is connected to terminals $X6-X5$ and $X6-X7$, respectively.

To observe the static current-voltage characteristics of the SPD in the closed state, the $S2$ switch connects
filter capacitor $C$, which provides power to the DC power circuit with a constant voltage.

To turn on the power supplies of the power circuit of the SPD it is necessary:

\begin{itemize}
\item set switch $S4$ and plug connector $SP1$ to the desired position (upper or lower);
\item set the autotransformer $T1$ to the lower position;
\item turn on the switch $S1$ and the power supply of the control circuit;
\item by pressing the $SB$ button, turn on the relay $K2$, 
the contacts of which close the power supply circuit of the SPD.
\end{itemize}

The voltage relay $K3$ prevents the relay $K2$ from turning on when there is voltage at the output 
of the autotransformer $T1$.
The sensitive current relay $K1$ traps the relay K2 in case of a dangerous increase in the leakage current 
of the SPD preventing it damage. When the connector $SP1$ is switched, 
the power supply circuit of relay coil $K2$ opens, therefore
After switching the connector, you must repeat its inclusion with the SB button.

\begin{landscape}
        \hspace{-3cm}
        \vspace{1cm}
\begin{figure}[ht!] 

%\begin{landscape}
%	\hspace{-3cm}
%	\vspace{1cm}
%	\begin{figure}[ht!]
\begin{circuitikz}[scale=0.48]
  \draw[color=black, thin]
(0,0) to [short,*-](-5,0) 
	(0,0) to [D-,l=$_{\small VD_{10}}$] (0,6)
	(0,6) to [D-, l=${\small VD_6}$, *-] (3,6) to [ammeter, l=${\small P\!A_2}$, *-*](7,6) to [short] (9,6) to [short] (9,8)
(3,6) to[short,-o] (3,5) node[below] {\small $X_8$}
	(6,6) to[short,-o] (6,5) node[below] {\small $X_{10}$}
(9,8) to[Ty,l_=$T$,*-o] (12,8) to [short,-o] (13,8) to [short] (14,8.5)
	(-5,0) to [D-, l=${\small VD_9}$, *-] (-5,2) to [D-, l=${\small VD_8}$] (-5,4) to [D-, l=${\small VD_7}$] (-5,6)
	(-5,6) to [D-, l=${\small VD_5}$, *-] (0,6)
	(7,6) to [voltmeter, l=${\small P\!V_2}$, *-*] (7,0) 
(0,0)--(7,0)--(14,0) to [short,-o] (14,7.8)
(14,12) to [short, -o] (14, 8.7) 
(12,8) -- (12,1) -- (22,1)
	(28,1) to [short, *-] (22,1) to [voltmeter, l_=${\small P\!V_3}$, *-*] (22,6)
(28,1) to [R, l_=$R_1$] (28,5) to[short] (28,6) to [short, -o] (29.5, 6) 
(28,1) to [short, -o] (29.5,1)
	(18,6) to [ammeter, l_=${\small P\!A_3}$, *-] (22,6) to [short, -*] (24.6,6) to [short] (24.6, 4) %A3
(24.6, 4) -- (25.4, 4)
(24.6, 6) -- (25.8, 6) to [R, l=$R_2$] (25.8, 3) to [short] (25.8, 2.3)
(18,6) -- (18,10) -- (11.2,10) to [short, -o] (11.2, 8.5)   %Упр к тиристору
(-11,5) node[transformer core] (T) {}   %%Трансформатор 2
(T.B1) -- (-8.9, 5)--(-8.9, 6) 
(T.B2) -- (-8.9, 2.6)--(-8.9, 0)
(-5,6)--(-8.9,6)
(-5,0)--(-8.9,0)
(T.base) node[above]{T3}
(-13.3, 4.8)-- (-13.1, 5) -- (-13.3, 5.2)
(T.A1)--(-13.1, 5)
(-13.5, 4.8)--(-13.3, 5)--(-13.5, 5.2)
(-13.5, 5) -- (-13.8, 5)
(-13.8, 5.2) -- (-14, 5) -- (-13.8, 4.8)
(-14, 5.2) -- (-14.2, 5) -- (-14 ,4.8)
(-14.2, 5)--(-14.5, 5)
(-15,0.2)--(-15, 0.8)--(-16,0.8) to [L,] (-16, 4.4)--(-16,7.8) %Дроссель
(T.A2)--(-13.5, 2.6)--(-13.5,0.8)--(-14.3,0.8)--(-14.3, 0.2)
(-14.5,0) node[below]{\small K2.4}
(-11,12) node[transformer core] (T) {} %%ЬТрансформатор 1
	(T.B1) to[D-, l=${\small VD_1}$](-7,12) to[D-, l=${\small VD_2}$](-5,12)to[D-, l=${\small VD_3}$](-1,12)--(0,12)--(0,12.2)--(1.2,12.2)
(1.2,12.2)--(1.2,11.8)--(0,11.8)--(0,12)
(T.B2) -- (-9.8, 7.8)--(-8.9, 7.8)--(-8.9, 7.2)
(-5, 7.8)--(-8.3, 7.8)--(-8.3, 7.2)
	(-8.4, 7.1) node[below]{\small K2.3}
(-13.3, 11.8)-- (-13.1,12) -- (-13.3, 12.2)
(T.A1)--(-13.1,12)
(-13.5, 11.8)--(-13.3, 12)--(-13.5, 12.2)
(-13.5, 12) -- (-13.8, 12)
(-13.8, 12.2) -- (-14, 12) -- (-13.8, 11.8)
(-14, 12.2) -- (-14.2, 12) -- (-14, 11.8)
(-14.2, 12)--(-14.5, 12) to [short, *-] (-14.5, 5)
(-14.5, 12) to [short, -*] (-16,12) -- (-18, 12) -- (-18, 9.4) 
(-16,12)--(-16,11)--(-16.6,11)--(-16.6, 10.6)--(-15.4,10.6)--(-15.4,11)--(-16,11)node[below right]{\small{K3.1}}
(-16,10.6) to [short, -*] (-16,7.8) -- (-12.6,7.8)--(-12.6, 9.6)--(T.A2)
(-16,7.8) to [short, *-*] (-19.5, 7.8) -- (-19.5, 8.3) to [L] (-19.5, 10.5)
(-19, 10.5)--(-19, 11.5)--(-21 ,11.5) to [short, -o](-21,11.9)--(-21.5, 12.9) % kluch B1
(-21,13) to [short, o-] (-21,13.6) node[below right]{В1}
(-21.3, 12.5)--(-22.5, 12.5)
(-21.2,12.3)--(-22.4, 12.3)
(-19.5, 7.8) -- (-22.2,7.8) to [fuse, -o] (-22.2, 11.9) --(-22.7, 12.9)
(-22.2, 13)  to [short, o-] (-22.2, 13.6)
	(T.base) node[above]{\small{T2}}
	(-5, 7.8) to [D-, l=${\small VD_4}$, *-*] (-5,12)
(-5, 7.8) to [short,-*] (-1, 7.8) -- (-1,8.4) to [nos,l_=$B$, o-o] (-1, 9.6) to [C,l_=$C_1$, -*] (-1,12)
(-1, 7.8) to [short, -*] (2.5, 7.8) to [voltmeter, l_=$V_1$, -*] (2.5, 12)--(1.2,12) %V1
(2.5, 7.8) -- (5.5, 7.8) -- (5.5, 8.2) -- (5.3, 8.4)
(5.5, 8.2) -- (5.7, 8.4)
(5.7, 8.6) --(5.5, 8.4) -- (5.3, 8.6)
(5.5, 8.6) -- (5.5, 9.4) to [short, -o] (7, 9.4) to [short, -o] (10.2, 10.4)
(2.5, 12) to  [ammeter, l=${\small P\!A_1}$, -*] (5.5, 12) -- (5.5, 11.6) -- (5.3, 11.4)
(5.5, 11.6) -- (5.7, 11.4) 
(5.3, 11.2)--(5.5, 11.4) -- (5.7, 11.2)
(5.5, 11.2) -- (5.5, 10.4) to [short, -o] (7, 10.4) to [short, -o] (10.2 ,9.4)
(9.6, 10.7)--(9.6, 9.7)
(9.8, 10.8)--(9.8, 9.8)
(-19.5, 10.5)--(-19, 10.5)
;
\draw[thick, ->](-18, 9.4) to [short] (-19.5, 9.4); %AT
\draw[thick, ->](24.6, 4) to [short] (25.4, 4); 
\draw[thick, ->](25.8, 2.3) to [short] (27.6, 2.3);
\draw[color=black, thick]
(9, 10.4) -- (10, 10.9)
(9, 9.4) -- (10, 9.9)
(9,8) to [short, -o] (9, 9.4)
(9, 10.4) to [short, o-] (9, 12) -- (14,12)
(29.5, 6) node[above] {+} node [below] {\small{15V}} (29.5,1) node[below]{-}
(-20.2, 8.3) -- (-20.2, 10.5) node[left] {\small{T1}}     
(-20.9, 13.7) node[above left] {\small{220V}}
(-22.2, 10.4) node[above left] {F}
(0.6,12.5)node[above]{\small K1.1}
(10.3, 10.9) node[right]{К1}
(14.4, 8.3)node[right]{К2}
(-16.3,2.5)node[left]{L}
     ;
\end{circuitikz}
%\label{LaboratoryArrangement}
%\caption{Laboratory arrangement}
%\end{figure}
%\end{landscape}
 
\caption{Laboratory arrangement}
\label{labsetup}
\end{figure}
\end{landscape}


\subsection{Task 1}

Study of thyristor characteristics

\begin{enumerate}
\item Connect the test thyristor $VS1$ to terminals $X1-X4$.
\item Observe and draw the static current-voltage characteristic of the thyristor control circuit 
$I_{ctrl} = f(U_{ctrl})$.
\item Determine the values of the control unlocking current $I_{ctrl.о}$ and the unlocking control voltage $U_{ctrl.o}$.
\item Observe and draw a direct branch of the classification current-voltage characteristic of the thyristor 
$I_a = f(U_a)$ in the open state.


Sketch from the oscilloscope screen a direct branch of the dynamic current-voltage characteristic of the thyristor
in the open state $I_a=f(U_a)$ and the curves of the anode current $I_a=f(\omega t)$ 
and voltage $U_a = f(\omega t)$ at maximum anode current.

\item Determine the pulse voltage in the open state, the value of the threshold voltage $U_0$
and differential resistance $Rd$.

\item observe and draw the direct and inverse branches of the classification 
$I_{a.av} = f (U_{a\: max})$ and static $I_{a.st} = f (U_{a.st})$
current-voltage characteristics of the thyristor in the closed state at the following current values
control: $I_{ctrl} = 0$; $I_{ctrl} = 0.9 I_{ctrl\:o}$; $I_{ctrl} = 2.0 I_{ctrl\:o}$.

In the classification scheme, draw from the oscilloscope screen
a form of voltage curves $U_a = f(\omega t)$, forward and reverse leakage currents $I_a = f(\omega t)$
at $I_{ctrl} = 0.9 I_{ctrl\:o}$ and the maximum possible values of the anode voltage.

\item Determine the thyristor class and the values of the direct ($I_{direct\:\:leak}$) and 
reverse ($I_{reverse\:\:leak}$) leakage currents.

\item Determine the value of the holding current of the thyristor $I_{hold}$.
\end{enumerate}

\subsection{Methodological instructions to accoplish the task}

\begin{enumerate}
\item The static current-voltage characteristic of the control circuit of the SPD is observed on a direct current
in the absence of the main (anode) voltage.

\item The values of the unlocking current and unlocking voltage control are determined in the circuit
of the SPD in the open state.

For this, it is necessary to set the average value at zero value of the control current.
the main (anode) voltage ($PV2$) 1.5 V and, gradually increasing the current and control voltage,
fix their values at the moment of the appearance of the main (anode) current (according to $PA$2).

\item Classification current-voltage characteristics of the power circuit of the SPD are observed at average values
half-wave voltage and current according to the readings of $PV2$ and $PA2$ devices at a control current equal to
1.2 $I_{ctrl\:\:o}$. Since the rated (classification) current of the tested SPD under natural cooling
equal to 6 A, the observing characteristics from 6 to 10 A in order to avoid overheating of the semiconductor structure
need to produce fast.

\item The observation of the direct branch of the dynamic current-voltage characteristics of the SPD 
in a state of high conductivity
produced from the screen of the oscilloscope when a signal 
is fed “vertically” to the input of the oscilloscope amplifier,
proportional to the current of the SPD and to the amplifier 
input "horizontally" - a signal proportional to the voltage
on the thyristor. To do this, turn off the oscilloscope sweep generator, connect a common point
amplifiers to terminal $X10$, and the inputs of amplifiers to terminals $X9$ and $X8$, respectively. 
Since the amplitude
sinusoidal half-wave current ($I_{a\:max}$) is in $\pi$ times more than its average value, 
the greatest deviation of the current curve on the oscilloscope screen corresponds 
to an instantaneous value equal to $I_{a\:\:max} = \pi I_{PA2}$,
where $I_{PA2}$ is the current value measured by the $PA2$ ammeter. 
The scale of the voltage curve is determined using external source. 
The curves drawn from the screen onto the tracing paper must be transferred to the graphs 
taking into account
their corresponding quadrants and scales.

Differential resistance and threshold voltage are determined by the dynamic current-voltage
characterization of open-loop SPD by approximation by its broken line, consisting of
a horizontal segment and an inclined beam crossing the characteristic at points $0.5 I_{m\:\: nom}$ and 
$1.5 I_{m\:\: nom}$,
where $I_{m\:\: nom} = 6\: A$ is the amplitude of the rated current STP.

\item To avoid damaging of the device when taking the current-voltage characteristics in the closed state
the applied voltage is increased smoothly, carefully fixing the increment of the leakage current and stopping
further increase in voltage, as soon as at a certain value of $U_{zigzag}$
a sharp increase in current begins.

When plotting the classification characteristics of the SPD in the closed state, the average value
voltage ($U_{a\:\:av}$), observed by the voltmeter $PV1$, is converted into the amplitude ($U_{a\:\: max}$)
 and characteristics
are constructed in the form of dependences $I_{a\:\:av} = f(U_{a\:\:max})$. 
Since the voltage has a half-wave sinusoidal shape,
conversion factor is $\pi$. Taking into account the same coefficient, 
the scale of the curves is determined by the voltmeter $PV1$
voltage $U_a = f (\omega t)$ observed when the oscilloscope is connected to terminals $X6-X5$. 
The scale of the current curves is $i_a = f (\omega t)$,
observed when connecting the oscilloscope to terminals $X6-X7$, is determined using the ammeter $PA1$ by the deviation
beam in the circuit for taking static characteristics, i.e. when $S2$ is on. Curve taking $U_a = f (\omega t)$ and
$I_a = f (\omega t)$ is produced at voltages close to $U_{zigzag}$.

\item The rated voltage of the SPD is determined as follows:

\begin{enumerate}
\item according to the current-voltage characteristic in the low conductivity state taken at $I_{ctrl} = 0$, 
the amplitude voltage values $\pm U_{zigzag}$;

\item the smaller of the stresses / + Usag / and / -Usag / Multiplied by the safety factor, the numerical value of which
usually around 0.75;

\item the obtained voltage value is divided by 100, and the result is rounded down to the nearest
integer, which is the class of a given valve, i.e., its rated voltage, expressed
in hundreds of volts.
\end{enumerate}

If the class turns out to be less than the third, for a more complete use of devices, division into classes
produced in 0.5.


\item The holding current of the SPD is determined in the circuit designed for observing the direct 
branch of the static volt-ampere
characteristics in a state of low conductivity. For this, at zero anode voltage
the control current is set equal to $1.5 I_{ctrl\:\:o}$; by adjusting the autotransformer $T1$, 
the anode current is set
at the maximum mark of the ammeter scale $PA1$, after which the control current decreases to zero and with a smooth
reducing the anode current by adjusting $T1$, its value ($I_{ctrl\: d}$) is fixed at the moment 
of switching off the SPD.
\end{enumerate}

\subsection{Report content}

The report should include:
\begin{enumerate}
\item laboratory setup and its brief description;
\item graphs and oscillograms specified in the task;
\item the conclusions based on research results.
\end{enumerate}
\end{document}
