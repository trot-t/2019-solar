\documentclass[a4paper,12pt]{article}
\usepackage{extsizes}

\usepackage[T2A]{fontenc}
\usepackage[utf8]{inputenc}
\usepackage[english,russian]{babel}
\usepackage{tikz}
\usepackage[european,cuteinductors,smartlabels]{circuitikz}

\usepackage{amsmath,amsfonts}
\usepackage{amssymb}
\usepackage[scr]{rsfso}



%%% Межстрочный интервал
\usepackage{setspace}

%% таблицы
\usepackage{booktabs}

%% для кода
\usepackage{color}
\usepackage{listingsutf8}

\definecolor{lightgrey}{rgb}{0.9,0.9,0.9}
\definecolor{lightblue}{rgb}{0,0,1}

\definecolor{grey}{rgb}{0.5,0.5,0.5}
\definecolor{blue}{rgb}{0,0,1}
\definecolor{violet}{rgb}{0.5,0,0.5}

\definecolor{darkred}{rgb}{0.5,0,0}
\definecolor{darkblue}{rgb}{0,0,0.5}
\definecolor{darkgreen}{rgb}{0,0.5,0}


\lstset{%
  language=C++,%
  morekeywords={constexpr,nullptr,size_t,uint32_t,assert,override,final},%
  basicstyle=\ttfamily\footnotesize,%
  sensitive=true,%
  keywordstyle=\color{blue},%
  stringstyle=\color{darkgreen},%
  commentstyle=\color{violet},%
  showstringspaces=false,%
  tabsize=2,%
  frame=leftline,
  rulecolor=\color{lightblue},
  xleftmargin=10pt,
}

\lstset{
%extendedchars=\true,
%inputencoding=utf8x,   
  numberstyle=\tiny,
  numbers=left,
  numbersep=10pt,
  xleftmargin=10pt,
  %framesep=4.5mm,
  %framexleftmargin=2.5mm,
  framexleftmargin=5pt,
  framesep=15pt,
  fillcolor=\color{lightgrey},
}


\title{Повышающий преобразователь}
\author{}
% Конец преамбулы
\begin{document}
%\maketitle
%\begin{tikzpicture}
%\newcommand{\xb}{-3}
%\newcommand{\xa}{3}
%\draw[thin, ->] (-6,0) -- (6,0) node[right] {$X$};
%\draw[thin, ->] (0,-6) -- (0,6) node[left] {$Y$};
%\foreach \x\xtext in {-5/-5,5/5,{\xb}/\xb,{\xa}/{\displaystyle \frac{-b+\sqrt{b^2-4ac}}{2a}}} % 
%   \draw (\x,0.1) -- (\x,-0.1) node[below] {$\xtext$};

 %\draw[domain=-5:5, help lines, smooth]
 %       plot ({\x},{0.2*(\x-\xa)*(\x-\xb)});
%\end{tikzpicture}


%\section{}

\begin{tikzpicture}
	\draw(0,3) to[L,l=L,o-] (3,3) -- (4,3) to [C,l_=C] (4,0) to[short,-o] (0,0);
	\draw(4,3) to[short,*-] (6.5,3) to[R,l_=R] (6.5,0) to [short,-*] (4,0);
	\draw[->,>=latex] (7.2,2.1) -- (7.2,0.9) node[midway,right] {$i_1$};
	\draw[->,>=latex] (4.7,2.1) -- (4.7,0.9) node[midway,right] {$i_2$};
	\draw (0,1.5) node {$U_1$};
\end{tikzpicture}

At t=0 we have $i_1(0) = i_{10}$ and $i_2(0) = i_{20}$.
$$
U=const
$$
Lets get graphs for $i_1(t), i_2(t)$ from analitic solution of differencial equations for the circuit.

\begin{equation}
\left\{
\begin{array}{lll}
U_1 = L\frac{\partial i}{\partial t} + R i_1 & \Rightarrow & 0 = \frac{\partial^2 i}{\partial t^2} + R \frac{\partial i_1}{\partial t} \\
	U_1 = L\frac{\partial i}{\partial t} + \frac{1}{C}\int\limits_0^t i_2(\tau) d\tau & \Rightarrow & 0 = \frac{\partial^2 i}{\partial t^2} + \frac{1}{C} i_2
\end{array}
\right.
\label{initial}
\end{equation}

Из первого и второго уравнения системы (\ref{initial}) получаем:

\begin{equation}
R\frac{\partial i_1}{\partial t} = \frac{1}{C} i_2 \Rightarrow \frac{\partial i_1}{\partial t} = \frac{1}{RC} i_2 
\label{first}
\end{equation}

из первого уравнения системы (\ref{initial}):

$$
\frac{\partial i_1}{\partial t} + \frac{\partial i_2}{\partial t} = -\frac{R}{L} i_1 + \frac{U_1}{L} 
$$

%Сведем систему 2-х уравнений первого порядка к одному уравнению второго порядка (исключаем $i_2$):

%$$
%\frac{\partial i_1}{\partial t} + RC\frac{\partial^2 i_1}{\partial t^2} = -\frac{R}{L} i_1 + \frac{U_1}{L}
%$$

%Решение есть сумма общего решения однородного уравнения
%$$
%RC\frac{\partial^2 i_1}{\partial t^2} +  \frac{\partial i_1}{\partial t}  + \frac{R}{L} i_1 = 0
%$$
%и частного решения неоднородного (с ненулевой правой частью).







В этом уравнении заменим $\frac{\partial i_1}{\partial t}$ из (\ref{first}):

$$
\frac{\partial i_2}{\partial t} = - \frac{R}{L} i_1 - \frac{1}{RC} i_2 + \frac{U_1}{L}
$$

\begin{equation}
\frac{\partial}{\partial t}\!\left(\begin{array}{c}i_1\\[1.5mm]i_2\end{array}\right) =
	\left(\begin{array}{cc}0&\frac{1}{RC}\\[1.5mm] -\frac{R}{L}&-\frac{1}{RC}\end{array}\right)
		\left(\begin{array}{c}i_1\\[1.5mm]i_2\end{array}\right)
		+
	\left(\begin{array}{c}0\\[1.5mm]\frac{U_1}{L}\end{array}\right)
\end{equation}

Предполагая решение в виде $i(t) = e^{\lambda t}$ получаем характеристическое уравнение.

$$
\left(\begin{array}{cc}-\lambda & \frac{1}{RC}\\[1.5mm] -\frac{R}{L} & -\lambda -\frac{1}{RC}\end{array}\right) =0  \Rightarrow \lambda^2 + \lambda \frac{1}{RC} + \frac{1}{LC} =0
$$

$$
D = b^2 - 4ac = \frac{1}{(RC)^2} - \frac{4}{LC}
$$

$$
\lambda_{1,2} = \frac{-b \pm \sqrt{D}}{2a} = -\frac{1}{2RC} \pm \sqrt{\frac{1}{(2RC)^2}- \frac{1}{LC}}
= -\alpha \pm j\omega
$$

$$
i_1(t) = A\cdot e^{\lambda_1 t} + B\cdot e^{\lambda_2 t} + \frac{U_1}{R}
$$
$$
i_2(t) = \left(A\cdot \lambda_1e^{\lambda_1 t} + B\cdot \lambda_2e^{\lambda_2 t}\right)\frac{1}{2\alpha}
$$



При $t=0$ получаем:

$$
A + B + \frac{U_1}{R} = i_{10}
$$
$$
\lambda_1 \cdot A + \lambda_2 \cdot B = i_{20}\cdot2\alpha
$$



Для определения коэффициентов получаем матрицу
$$
\left(
\begin{array}{cc}
  1 & 1 \\[1.5mm] 
  \lambda_1 & \lambda_2 
\end{array}
\right)\left(
\begin{array}{c}
  A\\[1.5mm]
  B 
\end{array}  
\right)=\left(
\begin{array}{c}
	i_1(0) - \frac{U_1}{R}\\[1.5mm]
  i_2(0) \cdot 2\alpha
\end{array}
\right)
$$
решая в Reduce-algebra
\lstinputlisting[linerange={1-3}]{AB.red}
$$
\begin{array}{ccl}
	A&=&-\frac{i_1(0)\lambda_2}{\lambda_1-\lambda_2} + \frac{i_2(0)\cdot2\alpha}{\lambda_1-\lambda_2} + \frac{\lambda_2}{\lambda_1-\lambda_2}\frac{U}{R}\\[1.5mm]
	B&=&\frac{i_1(0)\lambda_1}{\lambda_1-\lambda_2} - \frac{i_2(0)\cdot2\alpha}{\lambda_1-\lambda_2} - \frac{\lambda_1}{\lambda_1-\lambda_2}\frac{U}{R}
\end{array}
$$

упростим 
$$
\lambda_1 - \lambda_2 = \sqrt{\frac{1}{(RC)^2} - \frac{4}{LC}} = j \sqrt{\frac{4}{LC} - \frac{1}{(RC)^2}} = 2j\cdot \mathcal{I}m = 2j\omega
$$




еще три упрощения:
$$
\frac{\lambda_2e^{\lambda_1t} - \lambda_1e^{\lambda_2t}}{\lambda_1-\lambda_2} = -e^{-\alpha t}
\left(\cos(\omega t) + \frac{\alpha}{\omega} \sin(\omega t)\right)
$$

$$
\frac{e^{\lambda_1t} - e^{\lambda_2t}}{\lambda_1-\lambda_2} = e^{-\alpha t}
\frac{\sin\omega t}{\omega}
$$

$$
\frac{\lambda_1e^{\lambda_1t} - \lambda_2e^{\lambda_2t}}{\lambda_1-\lambda_2} = -e^{-\alpha t}
\left(\cos(\omega t) - \frac{\alpha}{\omega}\right)
$$

%%%  i_1
$$
i_1(t) = \frac{i_1(0) - \frac{U}{R}}{\lambda_1-\lambda_2} \left( -\lambda_2e^{\lambda_1 \cdot t}  + \lambda_1e^{\lambda_2 \cdot t} \right)
+\frac{i_2(0)2\alpha}{\lambda_1 - \lambda_2} \left(e^{\lambda_1\cdot t} - e^{\lambda_2\cdot t}\right)  + \frac{U}{R}=
$$

$$
= e^{-\alpha t} \left[ \left(i_1(0) - \frac{U}{R}\right)
\left(\cos(\omega t) + \frac{\alpha}{\omega} \sin(\omega t)\right)
+ i_2(0) \frac{\sin(\omega t)}{\omega} \right]  + \frac{U}{R} =
$$

$$
= \frac{e^{-\alpha t}}{\omega} \left[ \left(i_1(0) - \frac{U}{R}\right)
\left(\omega\cos(\omega t) + \alpha \sin(\omega t)\right)
+ i_2(0) \sin(\omega t) \right]  + \frac{U}{R}
$$


%%% i_2
$$
i_2(t) = RC\left\{-\alpha e^{-\alpha t} \left[ \left(i_1(0) - \frac{U}{R}\right)
\left(\cos(\omega t) + \frac{\alpha}{\omega} \sin(\omega t)\right)
+ i_2(0) \frac{2\alpha\cdot\sin(\omega t)}{\omega} \right]\right. +
$$
$$
+ \left.e^{-\alpha t} \left[  \left(i_1(0) - \frac{U}{R}\right) 
\left(-\omega\sin(\omega t) + \alpha \cos(\omega t)\right) + i_2(0)2\alpha\cos(\omega t)
\right]\right\} =
$$

$$
=  e^{-\alpha t} \frac{1}{2\alpha} \left[ \left(i_1(0) - \frac{U}{R}\right)
\left(-\omega - \frac{\alpha^2}{\omega}\right)\sin(\omega t) + 
i_2(0)\cdot 2\alpha\left(\cos(\omega t) - \frac{\alpha}{\omega}\sin(\omega t)\right)
\right] 
$$

здесь заметим, что 
$$
\alpha^2 + \omega^2 = \frac{1}{(2RC)^2} + \frac{1}{LC} - \frac{1}{(2RC)^2} = \frac{1}{LC}; \;\;\; 
\frac{1}{2\alpha} = RC; \;\;\;\; \frac{\alpha^2+\omega^2}{2a} = \frac{R}{L}
$$


$$
i_2(t)=\frac{e^{-\alpha t}}{\omega}
\left[ -\left(i_1(0) - \frac{U}{R}\right)\frac{R}{L}\sin(\omega t) 
+ i_2(0) \frac{1}{2\alpha} \left(\omega \cos(\omega t) - \alpha\sin(\omega t)\right)
\right] =
$$

$$
\frac{e^{-\alpha t}}{\omega}
\left[ -\left(i_1(0) - \frac{U}{R}\right)\frac{R}{L}\sin(\omega t)
+ i_2(0) RC \left(\omega \cos(\omega t) - \alpha\sin(\omega t)\right)
\right]
$$

Введем новые переменные ${\displaystyle \tilde{i}_1(t) = i_1(t) - \frac{U}{R}}$,
${\displaystyle \tilde{i}(t) = i(t) - \frac{U}{R}}$

тогда:
$$
\tilde{i}_1(t) = \frac{e^{-\alpha t}}{\omega} \left[ \tilde{i}_1(0) 
\left(\omega\cos(\omega t) + \alpha \sin(\omega t)\right)
+ i_2(0)\cdot 2\alpha \sin(\omega t) \right]
$$
$$
i_2(t)=\frac{e^{-\alpha t}}{\omega}
\left[ -\tilde{i}_1(0) \frac{\alpha^2+\omega^2}{2\alpha}\sin(\omega t) +
i_2(0) \left(\omega \cos(\omega t) - \alpha\sin(\omega t)\right)
\right]
$$

$$
\tilde{i}(t) = \frac{e^{-\alpha t}}{\omega} 
\left[
	\tilde{i}_1(0)\left(\omega\cos(\omega t) + \frac{\alpha^2 - \omega^2}{2\alpha}\sin(\omega t) \right)
+ i_2(0)\left(\omega\cos(\omega t)  + \alpha\sin(\omega t)\right)
\right]
$$

% mathcal{Re}

%[  - i_10*l*lambda_2 + i_20*l + lambda_2*u ]
%[------------------------------------------]
%[         l*(lambda_1 - lambda_2)          ]
%[                                          ]
%[  i_10*l*lambda_1 - i_20*l - lambda_1*u   ]
%[ ---------------------------------------  ]
%[         l*(lambda_1 - lambda_2)          ]

Построить графики аналитического решения для токов $i_1(t)$, $i_2(t)$ и $i(t)$ с помощью программы {\bf gnuplot}. Пример построения в программе {\bf gnuplot}
представлен ниже:

\lstinputlisting{gnuplot.cmd}

Получить численное решение дифференциальных уравнений с помощью программы {\bf octave}. Пример численного решения в программе {\bf octave} представлен ниже:

\lstinputlisting{octave4.txt}

Смоделировать решение дифференциальных уравнений с помощью программы {\bf ngspice}. Пример моделирования решения с помощью {\bf ngspice} представлен ниже:

\lstinputlisting{check.cir}

Сравнить решения.

\begin{tikzpicture}
\draw(0,2) to [L,o-] (2,2) -- (2,0) to[short,-o] (0,0);
\draw ({4+0},0) to[C] ({4+0},2) -- ({4+2},2) to[R] ({4+2},0) -- ({4+0},0); 
\end{tikzpicture}

уравнение цепи на интервале $(n+\gamma)T \le t \le (n+1)T$

$$
RC \frac{d U_C}{dt} + U_C = 0
$$
Введем новую переменную -- относительное время
$$
\bar{t} = \frac{t}{T}
$$
Введем новую переменную $\beta^{\prime\prime} = \frac{T}{RC}$

тогда для уравнения
$$
\frac{1}{\beta^{\prime\prime}}\frac{dU_C}{dt} + U_C = 0
$$

получаем решение
$$
U_C(t) = B_c e^{-\beta^{\prime\prime}(\bar{t}-n-\gamma)}
$$
аналогично для тока через резистор
$$
i_1(t) = i_1(n+\gamma) e^{-\beta^{\prime\prime}(\bar{t}-n-\gamma)}
$$


для цепи с индуктивностью 
$$
U = L\frac{d i_L}{dt} 
$$
решение уравнения:
$$
i(t) = i(n+\gamma) + L\cdot t
$$


\subsection{Laplace way}

$$
\left\{
\begin{array}{lcl}
%	px - \frac{1}{RC} = 
	Li_1^\prime + L i_2^\prime + Rx &=& U\\[1.5mm]
	Li_1^\prime + L i_2^\prime + \frac{1}{C}\int i_2 d\tau &=& U
\end{array}
\right.
$$

$$
\left\{
\begin{array}{lcl}
	i_1^\prime &=& \frac{1}{RC} i_2\\[1.5mm]
	i_2^\prime &=& \frac{U}{L} - \frac{R}{L} i_1 - \frac{1}{RC} i_2
\end{array}
\right.
$$

Делаем замену
$$
\frac{U}{L} \doteqdot \frac{U}{L} \cdot \frac{1}{p};\; \; \; i_1 \doteqdot \tilde{I}_1 ;\; \; \; i_2 \doteqdot \tilde{I}_2; \; \; \;
i_1^\prime = \tilde{I}_1\cdot p - i_1(0); \; \; \;  i_2^\prime = p \tilde{I}_2 - i_2(0)
$$

$$
%\left\{
%	\begin{array}{lcl}
%		p\cdot \tilde{I}_1 - i_1(0) &=& \frac{1}{RC} \cdot \tilde{I}_2\\[1.5mm]
%		p\cdot \tilde{I}_2 - i_2(0) &=& \frac{U}{L}\cdot \frac{1}{p} - \frac{R}{L}\cdot \tilde{I}_1 -  \frac{1}{RC} \cdot \tilde{I}_2
%	\end{array}
%\right.
$$

\subsection{Дискретное преобразование Лапласа}

$$
i_1(n+1) = U_c(n+1)/R
$$
$$
i_2(n+1) = i(n+1) - i_1(n+1)
$$

$$
U_c(n+1) = U_c(n+\gamma) e^{-\beta^\prime (\bar{t} - n - \gamma)} 
$$
$$
i_1(n+1) = i_1(n+\gamma) e^{-\beta (1-\gamma)}
$$
$$
i(n+1) = i(n+\gamma) + L\cdot (1-\gamma)
$$

$$
U_c(n+\gamma) = R\cdot i_1(n+\gamma)
$$
%$$
%i_2(n+\gamma) = \frac{e^{-\alpha\gamma T}}{\omega}
%\left[ -\tilde{i}_1(n) \frac{\alpha^2+\omega^2}{2\alpha}\sin(\omega\gamma T) +
%i_2(n) \frac{1}{2\alpha} \left(\omega \cos(\omega\gamma T) - \alpha\sin(\omega\gamma T)\right)
%\right] 
%$$
$$
\tilde{i}(n+\gamma) = \frac{e^{-\alpha\gamma T}}{\omega} 
\left[
\tilde{i}_1(n)\left(\omega\cos(\omega\gamma T) + \frac{\alpha^2 - \omega^2}{2\alpha}\sin(\omega\gamma T) \right)
+ i_2(n)\left(\omega\cos(\omega\gamma T)  + \alpha\sin(\omega\gamma T)\right)
\right]
$$


Выразим $n+1$ через $n$:
$$
\tilde{i}[n+1] = L\cdot (1-\gamma) +
$$
$$
+ \frac{e^{-\alpha\gamma T}}{\omega}
\left[
\tilde{i}_1[n] \left(\omega\cos(\omega\gamma T) + \frac{\alpha^2 - \omega^2}{2\alpha}\sin(\omega\gamma T) \right)
+ i_2[n]\left(\omega\cos(\omega\gamma T)  + \alpha\sin(\omega\gamma T)\right)
\right] =
$$ 
$$
= L\cdot (1-\gamma) + \frac{e^{-\alpha\gamma T}}{\omega}
\left[
\tilde{i}_1[n]\left(\omega\cos(\omega\gamma T) + \frac{\alpha^2 - \omega^2}{2\alpha}\sin(\omega\gamma T) \right) \right. +
$$
$$
\left.
+ \left(\tilde{i}[n] - \tilde{i}_1[n]\right)\left(\omega\cos(\omega\gamma T)
+ \alpha\sin(\omega\gamma T)\right)
\right]
$$
%$$
%i_2(n+1) = i(n+1) - i_1(n+1) = i(n+1) - i_1(n+\gamma) e^{-\beta (1-\gamma)} =
%$$
%$$
%=i(n+1) - \left\{ \frac{e^{-\alpha t}}{\omega} \left[ \tilde{i}_1(0) 
%\left(\omega\cos(\omega t) + \alpha \sin(\omega t)\right)
%+ i_2(0) \sin(\omega t) \right] + \frac{U}{R} \right\}  e^{-\beta (1-\gamma)}
%$$
$$
\tilde{i}_1[n+1] +\frac{U}{R} = \left(\frac{e^{-\alpha\gamma T}}{\omega} 
\left[ \tilde{i}_1[n] 
\left(\omega\cos(\omega\gamma T) + \alpha \sin(\omega\gamma T)\right)
+ i_2[n] 2\alpha\sin(\omega\gamma T) \right] + \frac{U}{R}\right) e^{-\beta (1-\gamma)} =
$$ 
$$
= \frac{e^{-\alpha\gamma T -\beta(1-\gamma)}}{\omega} \left( \tilde{i}_1[n] 
\left(\omega\cos(\omega\gamma T) + \alpha \sin(\omega\gamma T)\right)
+ \left(\tilde{i}[n] - \tilde{i}_1[n]\right)2\alpha\sin(\omega\gamma T) \right)
+ \frac{U}{R} e^{-\beta (1-\gamma)}
$$

На основании теоремы сдвига:
$$
\left.
\begin{array}{lcl}
D\left\{i[n+1]\right\} &=& e^q I^*(q) - e^q i[0]\\[1.5mm]
D\left\{i_1[n+1]\right\} &=& e^q I_1^*(q) - e^q i_1[0]
\end{array}
\right\}
$$

$$
e^q I^*(q) - e^qi[0] - L\cdot(1-\gamma)\frac{e^q}{e^q - 1} = 
$$
$$
= \frac{e^{-\alpha\gamma T}}{\omega} \left[
-\tilde{I}_1^*(q) \frac{\alpha^2+\omega^2}{2\alpha} sin(\omega\gamma T) 
+ I^*(q) \left(\omega\cos(\omega\gamma T) + \alpha\sin(\omega\gamma T)\right)
\right]
$$

$$
e^q\tilde{I}^*_1(q) - e^q i_1[0] + 
\frac{U}{R}\left(1- e^{-\beta(1-\gamma)} \right) \frac{e^q}{e^q-1} =
$$
$$
= \frac{e^{-\alpha\gamma T - \beta(1-\gamma)}}{\omega} \left(
\tilde{I}_1^*(q) \left(\omega\cos(\omega\gamma T) + \alpha\sin(\omega\gamma T)\right)
+ \left(I^*(q) - \tilde{I}^*_1(q) \right)\cdot 2\alpha\sin(\omega\gamma T)
\right)
$$

получаем систему уравнений относительно $I(q)$ и $\tilde{I}_1(q)$:

$$
\hspace{-2cm}
\left(
\begin{array}{cc}
{\displaystyle \frac{e^{-\alpha\gamma T}}{\omega}(\omega\cos(\omega\gamma T) + \alpha\sin(\omega\gamma T)) - e^q} 
&
{\displaystyle -\frac{e^{-\alpha\gamma T}}{\omega}\frac{\alpha^2+\omega^2}{2\alpha}\sin(\omega\gamma T)}\\
{\displaystyle 2\alpha\frac{e^{-\alpha\gamma T - \beta(1-\gamma)}}{\omega}\sin(\omega\gamma T)}
&
{\displaystyle \frac{e^{-\alpha\gamma T - \beta(1-\gamma)}}{\omega}
	\left( \omega\cos(\omega\gamma T) - \alpha \sin(\omega\gamma T)\right)} - e^q
\end{array}
\right)\times
$$
$$
\times\left(
\begin{array}{c}
I^*(q)\\
\tilde{I}_1^*(q)
\end{array}
\right)=
\left(
\begin{array}{c}
{\displaystyle - e^qi[0] - L\cdot(1-\gamma)
	\frac{e^q}{e^q - 1}} \\[2.2mm]
{\displaystyle - e^q i_1[0] +
\frac{U}{R}\left(1- e^{-\beta(1-\gamma)} \right) \frac{e^q}{e^q-1}}
\end{array}
\right)
$$

\subsection{Предельное значение}

Предельное значение $t\rightarrow \infty$ определяется по формуле:

$$
\lim_{n\rightarrow \infty} (e^q -1) F^*(q)
$$

\end{document}
