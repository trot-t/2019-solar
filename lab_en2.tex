\documentclass[a4paper,14pt]{article}
\usepackage{lscape}
%%% Страница
\usepackage{extsizes} % Возможность сделать 14-й шрифт
\usepackage{geometry} % Простой способ задавать поля
        \geometry{top=25mm}
        \geometry{bottom=30mm}
        \geometry{left=30mm}
        \geometry{right=20mm}
 %

\usepackage[T1,T2A]{fontenc}
\usepackage[utf8]{inputenc}
\usepackage{amssymb,amsmath}
\usepackage{float}
\usepackage[unicode, pdftex]{hyperref}
\usepackage[europeanresistors,americaninductors]{circuitikz}
\usetikzlibrary{calc}
\usepackage[T1,T2A]{fontenc}
\usepackage[utf8]{inputenc}

\usepackage{amssymb,amsmath}
\usepackage{float}
\usepackage[unicode, pdftex]{hyperref}

\usepackage{tikz}
\usepackage{rotating}
%\usepackage[landscape]{geometry}
\usepackage{graphicx}
\graphicspath{{pictures/}}
\DeclareGraphicsExtensions{.pdf,.png,.jpg}
\usepackage{pgfplots}
\usepackage{wrapfig}
\usepackage{rotating}
\usepackage{lipsum}
\usepackage{nccmath}
\usepackage{caption}
\usepackage{siunitx}
%\usepackage[american,cuteinductors,smartlabels]{circuitikz}
%\usepackage[backend=biber]{biblatex}

\usepackage[]{hyperref}
\ctikzset{bipoles/thickness=1}
\ctikzset{bipoles/length=0.8cm}
\ctikzset{bipoles/diode/height=.375}
\ctikzset{bipoles/diode/width=.3}
\ctikzset{tripoles/thyristor/height=.8}
%\ctikzset{tripoles/thyristor/width=1}
\ctikzset{tripoles/thyristor/width=0.8}
\ctikzset{bipoles/vsourceam/height/.initial=.7}
\ctikzset{bipoles/vsourceam/width/.initial=.7}
\tikzstyle{every node}=[font=\small]
\tikzstyle{every path}=[line width=0.8pt,line cap=round,line join=round]

\ctikzset{resistor = european}
\ctikzset{inductor = american}
\ctikzset{tripoles/thyristor/height=0.55}

\ctikzset{bipoles/cuteindictor/coils/.initial=3}
\ctikzset{bipoles/americanindictor/coils/.initial=3}
\ctikzset{bipoles/cuteindictor/coils/.initial=3}
\ctikzset{bipoles/americanindictor/coils/.initial=3}
\hypersetup{
colorlinks=false,
}
\usepackage{textcomp}

\begin{document}


\section{Research of unmanaged rectifiers and filters}

The aim of the work is studying the characteristics of various options for uncontrolled circuits.
rectifiers of single and three-phase voltages and the influence of smoothing filters on their operation.

\subsection{Lab Description}

At the laboratory bench (Fig. \ref{labsetup2}) using the appropriate switch toggle switches
You can explore the following rectifier circuits:

\begin{itemize}
	\item single-phase half-wave;
	\item single-phase one-half-wave with a shunt diode;
	\item single-phase two-half-period with the deduced zero point on secondary winding 
		of the converter transformer;
\item single-phase bridge;
\item three-phase with zero output;
\item three-phase bridge.
\end{itemize}

To smooth the ripple of the load current at the output of the rectifiers, one of the following 
can be switched on filter options: inductive ($L$-filter), capacitive ($C$-filter), $L$-shaped ($CL$- and $LC$-filter) 
and $\Pi$-shaped ($CLC$ filter).

The rectifiers are powered by a $T4$ transformer connected to the network through $T1-TZ$ autotransformers,
by which the required value of the supply voltage is set. The whole lab bench is turned on by using $Q1$.

To obtain the desired rectifier circuit, it is necessary to put in the appropriate positions
$S1$, $S3$ and $S11-S16$ switches and switch $S17$. So, for example, to obtain a three-phase bridge circuit, 
it is necessary turn on $S1$ and $S11-S16$, and set the switch $S17$ to the upper position. 
By means of switches $S4-S6$ an additional inductance ($L1-L3$) can be included in the circuit of each phase 
of the AC rectifier current.

The type and parameters of the smoothing filter are selected by switches $S7-S10$ and switch $S19$.

The load resistance is changed by tunting the resistor $R3$ with switch $S18$ and regulation of resistance $R2$.

The alternating voltage at the input of the rectifier is measured with a voltmeter $PV1$, connected to each 
of the phases via switch $S2$.

With the help of ammeter $PA1$ and wattmeter $PW$, the current and power of one of the primary windings of 
transformer $T4$ are measured; the total active power consumed by the rectifier is proportional to 
the number of windings used transformer $T4$.

Voltmeter $PV2$ and ammeters $PA1$, $PA3$ are designed to measure the voltage and current of the secondary (valve) 
winding of transformer $T4$. Moreover, $PA2$ (electromagnetic system) measures the effective value, and $PA3$ 
-- the average value.
Similarly, to measure the voltage and load current, two pairs of instruments $PV4$, $PV5$ and $PA4$, $PA5$, 
are used, in each of which one determines the effective and the other the average value of the measured values.

\begin{landscape}
        \hspace{-3cm}
        \vspace{1cm}
\begin{figure}[ht!]
\begin{circuitikz}[scale=0.5]
  \draw[color=black, thin]
  (1,0) node[above right] {380В}
(-0.1,-1)--(2.9,-1)
(-0.2,-0.8)--(2.8,-0.8)
%(0,-1) to [short, -o] (0,0)
(0,-2) to [nos,-o] (0,0)
%(1.5,-1) to [short, -o] (1.5,0)
(1.5,-2) to [nos,-o] (1.5,0)
%(3,-1) to [short, -o] (3,0)
(3,-2) to [nos,-o, l_=$S1$] (3,0)
(0,-2) to [fuse, l=$F1$](0,-4) to [L, l=$T1$,mirror](0,-6)to [short, -*] (1.5,-6)--(3,-6)
(1.5,-2) to [fuse, l=$F2$] (1.5,-4)to [L, l=$T2$,mirror](1.5,-6)
(3,-2) to [fuse, l=$F3$] (3,-4)to [L, l=$T3$,mirror](3,-6)
(0,-5)--(0.2,-5)--(0.2,-8)
(0.2,-12.75) to [short,l=$~~~~~~S20$ ] (0.2,-13.25)
(0.2,-9) to [nos,l_=$S2$] (0.2,-8)
(0.2,-9) to [short, -*] (0.2,-11)--(0.2,-12)
(1.5,-5)--(1.7,-5) to [short, -*] (1.7,-9)--(1.7,-12)
(3,-5)--(3.2,-5) to [short, -*] (3.2,-7)--(3.2,-12)
(0.2,-11)--(7,-11)to [L,-*](9,-11)
(1.7,-9)--(7,-9)to [L,-*](9,-9)
(3.2,-7)to[ammeter, l=$PA1$, -*](5,-7) to node[mixer, rotate=45]{}(7,-7) to [L,*-, l=$~~~~~~~~~~T4$](9,-7)
(4,-13)--(-0.5,-13)--(-0.5,-14) to[voltmeter, l=$PV1$,] (9,-14)--(9,-7)

(12,-11)--(12,-10) to [nos,l=$S6$] (14,-10)--(14,-11)
(12,-9)--(12,-8) to [nos,l=$S5$] (14,-8)--(14,-9)
(12,-7)--(12,-6) to [nos,l=$S4$] (14,-6)--(14,-7)
(9.5,-5)to [voltmeter, l=$PV2$,*-](14,-5)to [short,-*] (14,-6) 

(9.5,-11)to [L,*-*](12,-11) to [L,-*,l=$L3$](14,-11)--(21,-11)
(9.5,-9)to [L,*-*](12,-9)   to [L,-*,l=$L2$](14,-9)--(19,-9)
(9.5,-7)to [L,*-*](12,-7)   to [L,-*,l=$L1$](14,-7)to [ammeter, l=$PA2$](15.5,-7)to[ammeter, l=$PA3$](17,-7)

(21,-13.5)to [nos,l_=$S12$](21,-12)to[D-,l_=$V2$,-*] (21,-11)--(21,-6)to[D-,l_=$V5$] (21,-5) to [nos,l_=$S15$,-*](21,-3)
(19,-13.5)to [nos,l_=$S16$](19,-12)to[D-,l_=$V6$] (19,-11)to[short, -*] (19,-9)--(19,-6)to[D-,l_=$V3$] (19,-5)to [nos,l_=$S13$](19,-3)
(17,-13.5)to [nos,l_=$S14$](17,-12)to[D-,l_=$V4$] (17,-11)to[short, -*] (17,-7)--(17,-6)to[D-,l_=$V1$] (17,-5)to [nos,l_=$S11$](17,-3)

(21,-12.5)--(19,-12.5)to[short,*-](17,-12.5)

(21.5,-14)--(21.5,-13.5)to [short, -*] (21,-13.5)to [short, -*] (19,-13.5)-- (17,-13.5)
(21.5,-14.5)--(21.5,-15)--(9.5,-15)--(9.5,-3)to[D-,l=$V7$] (11,-3)to [nos,l=$S3$,-*](17,-3)--(19,-3)to [short, *-] (25,-3)--(26.5,-3) to[L, l=$L$](28,-3)

(21.5,-14.25)to[short,-*, l_=$S17$](22.5,-14.25)--(24,-14.25)to [short, *-*] (25.5,-14.25)to [short, *-*] (27,-14.25)to [short, *-*] (28.5,-14.25)to [short, *-*] (30,-14.25)to [short, *-*](31.5,-14.25)to [short, *-](33,-14.25)

(24,-14.25)to [C,l_=$C1$](24,-12)to[nos,-*,l_=$S7$](24,-8)to[short,*-*](24,-3)
(25.5,-14.25)to [C,l_=$C2$](25.5,-12)to[nos,l_=$S8$](25.5,-8)--(21,-8)

(27,-14.25)to [C,l_=$C3$](27,-12)to[nos,-*,l_=$S9$](27,-8)to[short,*-*](27,-5)
(28.5,-14.25)to [C,l_=$C4$](28.5,-12)to[nos,l_=$S10$](28.5,-8)--(27,-8)

(30,-14.25)to [voltmeter, l_=$PV4$,*-](30,-9)--(30,-6)to [short,-*] (31.5,-6) 
(31.5,-14.25)--(31.5,-12)to [voltmeter, l=$PV5$](31.5,-6)to [short,-*] (31.5,-6) --(31.5,-5)

(32.25,-14.25)to [nos,l=$S18$,mirror](32.25,-11)--(33,-11)

(33,-14.25)to[R,-*,l_=$R3$](33,-11)to[R,-*,l_=$R2$](33,-9)to[R,l_=$R1$](33,-5)to[short,-*](31.5,-5)
(28.5,-3.5)--(26,-3.5)--(26,-5)--(27,-5)to[ammeter, l=$PA4$](29,-5)to[ammeter, l=$PA5$](31.5,-5)
;

\draw[color=black, thick]
(26.5,-2.7)--(28,-2.7)
(-0.3,-4.5) -- (-0.3,-5.5)
(1.2,-4.5) -- (1.2,-5.5)
(2.7,-4.5) -- (2.7,-5.5)
(9.25,-7) -- (9.25,-11)
;
\end{circuitikz}

\caption{Laboratory arrangement}
\label{labsetup2}
\end{figure}
\end{landscape}

\subsection{The task}

\begin{enumerate}
	\item For rectifier schemes specified by the teacher, smoothing filter parameters and input variable values
of AC voltage calculate and determine experimentally:

\begin{itemize}
\item the relationship between the values of voltages ($E_{d0}/E_{2ph.rms}$) 
	and currents ($I_{2ph.rms} / I_d$, $I_{2ph.rms.avr.} / I_d$)  at the input and output rectifier circuits, 
where $E_d0$ is the rectified EMF of the rectifier; $E_{2ph.rms}$ - phase EMF of the valve winding
transformer; $I_{2ph.rms}$ and $I_{2ph.rms.avg}$ -- current and average current values of the phase winding 
of the transformer; $I_d$ is the average value of the rectified current;

\item external characteristics $U_d = f (I_d)$;

\item the dependence of the smoothing coefficients of the filters from the load current 
	$K_{smooth} = f (I_d)$.
\end{itemize}

	Dependencies $U_d = f (I_d)$ and $K_{smooth} = f (I_d)$ are presented in the form of graphs.

\item Draw from the oscilloscope screen the shapes of the curves of rectified voltages and currents 
of the investigated options of rectifier circuits and smoothing filters.

\item For a single-phase rectifier circuit specified by the teacher with and without inductive filter, determine
experimentally and construct the energy characteristics $\eta = f (I_d), \lambda = f (I_d)$. Rectifier efficiency
determine by indicators of instruments and current and average values of rectified voltages and currents.
\end{enumerate}

\subsection{Methodological instructions for the performance of work}

\begin{enumerate}
\item The calculated value of the relationship between rectified and phase voltage for half-wave rectifier circuits
for $m> 1$ is determined by the expression

$$
		\frac{E_{d0}}{E_{2ph.rms}} = \frac{m}{\pi} \sqrt{2}\sin\frac{\pi}{m}
$$
where $E_{2ph.rms}$ is the effective value of the phase voltage of the valve windings of the transformer; 
$m$ is the number of rectifier phases (number of ripples over the network period).

For full-bridge rectifier circuits, the value of this ratio is twiced.

\item The calculated values of the ratios of average and effective values of phase currents to rectified current for
half-bridge rectifier circuits (for $m> 1$) are:
$$
  \frac{I_{2ph.rms\:\:avr}}{I_d} = \frac{1}{m}; \frac{I_{2ph.rms}}{I_d} = \frac{1}{\sqrt{m}}.
$$

For full-bridge rectifier circuits, these relations are:
$$
    \frac{I_{2ph.rms\:\:avr}}{I_d} = 0; \frac{I_{2ph.rms}}{I_d} = \frac{2}{\sqrt{m}}.
$$		


\item The experimental values of the ratios $E_{d0} / E_{2ph.rms}$ should be determined at maximum values
rectified current according to the voltmeter of the magnetoelectric system, and the experimental values
ratios of currents should be determined at maximum values of the inductance of the $L$-filter and the load current.

\item The calculation of the external characteristics of the zero rectifier circuits with active-inductive load 
should be performed according to the equation
$$
		U_d = E_{d0} -I_d\left( R_{tr} + R_{dyn.valve} + R_{reactor} + R_{comm}\right) - U_0
$$

where $R_{tr}$ is the total resistance of the transformer phase, reduced to its secondary winding:
$R_{reactor}$ -- active resistance of the inductor winding, $R_{dyn.valve}$ -- dynamic resistance of the valve;
		${\displaystyle R_{comm}=\frac{m\xi_a}{2\pi}}$ 
-- switching resistance; $\xi_a$ is the reduced inductive reactance of the transformer;
$U_0$ is the threshold voltage of the valve.

For full-bridge rectifier circuits, the values of $R_{tr}$, $R_{dyn.valve}$, $R_comm$ and $U_0$ will doubled.

Experimental external characteristics are measured by the average voltage values
and current for all given variants of rectifier and filter circuits.

Required to calculate the parameters of the rectification scheme are given in the table located on the front panel 
of the stand.

\item The calculated value of the smoothing coefficient for the inductive filter is determined by the expression
$$
		k_{smooth} = \frac{2\pi f_p L}{R_{load}}
$$


where $f_p$ is the ripple frequency of the rectified voltage; $L$ is the inductance of the filter;
$R_{load}$ is the load resistance of the rectifier.

The value of the smoothing coefficient of the $L$-shaped $LC$ filter can be determined by the formula

$$
K_{smooth} \cong (2\pi f_p)^2 LC
$$

where $C$ is the filter capacity.

For an experimental study of pulsations, an oscilloscope can be used.
\end{enumerate}

\subsection{The content of the report}

The report should include:
\begin{enumerate}
	\item scheme of the experimental setup and its brief description;
	\item calculations of given parameters and characteristics $K_{smooth} = f(I_d)$;
	\item tables of experimental observations;
	\item graphs Ud u003d f (Id) obtained by calculation and experimentally;
	\item waveforms of rectified voltage curves;
	\item conclusions comparing the results of theoretical and experimental studies.
\end{enumerate}
\end{document}
